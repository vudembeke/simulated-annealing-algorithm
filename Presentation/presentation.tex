\documentclass{beamer}
\usepackage{graphicx} % Required for inserting images
\usetheme{default}

\title{Simulated Annealing Algorithm}
\author{VUDEMBEKE BRIANNA, NYAGA MUTHOMI NATHANIEL}
\date{March 25, 2023}

\begin{document}

\begin{frame}
\titlepage
\end{frame}

\begin{frame}{Introduction}
\begin{itemize}
\item Simulated annealing is a probabilistic technique for finding global optimum.
\item This algorithm is a proposed alternative to generic greedy search, hill climbing algorithms.
\item Hill climbing algorithms are good at finding local optimal solutions but don't perform well with global solutions.
\item Simulated annealing is inspired by the process of annealing in metallurgy where a material is heated and slowly cooled under controlled conditions to increase the size of the crystals in the material and reduce their defects.
\end{itemize}
\end{frame}

\begin{frame}{Simulated Annealing Algorithm}
\begin{enumerate}
\item Generate an initial candidate solution x.
\item Get an initial Temperature $ T > 0$.
\item for i in 1:N(N = number of iterations)
\begin{itemize}
\item Sample $ \zeta ~ g(\zeta)$ where g is a symmetrical distribution.
\item The new candidate solution is $x' = x ± \zeta$
\item Calculate probability $p=exp(\Delta h/T_{i})$
\item Accept the candidate solution with probability $p; u U(0, 1)$, accept $x = x'$ if $u \leq p.$
\item Update the temperature (cooling), e.g. $T = aT$ where $0 < a < 1$
\end{itemize}
\end{enumerate}
\end{frame}

\begin{frame}{Simulated Annealing Algorithm Contd.}
\begin{itemize}
\item The main difference in strategy between greedy search and simulated annealing is that greedy search will always choose the best proposal, whereas simulated annealing has a probability (using a Boltzmann distribution) of choosing a worse proposal than strictly only accepting improvements.
\item This helps the algorithm find a global optimum by jumping out of local optimum.
\end{itemize}
\end{frame}

\begin{frame}{Notes on Parameters}
\begin{itemize}
\item The greater the value of T (temperature), the greater the likelihood of moving around the search space. As T gets closer to zero, the algorithm will function like greedy hill climbing.
\item Good starting values for T will vary problem by problem. We usually start with 1, 10, 100, and adjust after a few experiments.
\item For a, we normally choose 0.95. However, you can change T by any amount, Robert and Casella suggest a temperature decrease in $1/\log(1+i)$ for $i$ in $1:N$.
\item For a review of cooling schedules, we recommend reading "A comparison of simulated annealing cooling strategies."
\end{itemize}
\end{frame}

\begin{frame}{References}
\begin{itemize}
\item Introduction to Monte Carlo Methods with R by Robert and Casella.
\item Optimization by Simulated Annealing: An Experimental Evaluation; Part I, Graph Partitioning.
\end{itemize}
\end{frame}

\end{document}